\documentclass{report}

\begin{document}

\chapter*{Project Description Statement}

\section*{Descrizione del Progetto:}
Il progetto mira a sviluppare un avanzato marketplace energetico online che agevoli la compravendita di energia rinnovabile. La piattaforma offrirà un'interfaccia intuitiva per permettere a produttori di energia, consumatori e fornitori di connettersi, negoziare e completare transazioni in modo efficiente e trasparente.


\section*{Obiettivi}
\begin{itemize}
    \item Sviluppo di una piattaforma web-based intuitiva
    \item Connessione efficace tra venditori e utenti consumatori
    \item Promozione efficace dei prodotti energetici
    \item Ottimizzazione delle risorse energetiche
    \item Scalabilità e adattabilità
\end{itemize}

\section*{Attività e Fasi:}
\begin{enumerate}
    \item Sviluppo della Piattaforma: Progettazione e sviluppo della piattaforma web, inclusa l'interfaccia utente e le funzionalità di registrazione, autenticazione e gestione del profilo.
    \item Integrazione degli Utenti: Creazione di profili per i produttori di energia, i consumatori e i fornitori, consentendo loro di accedere alle funzionalità del marketplace.
    \item Implementazione di Funzionalità di Compravendita: Sviluppo di strumenti per la pubblicazione di offerte e richieste di energia, nonché per la gestione delle trattative e delle transazioni.
    \item Monitoraggio e Ottimizzazione: Integrazione di strumenti di monitoraggio per tracciare le attività sulla piattaforma e ottimizzarne l'efficienza.
    \item Test e Validazione: Esecuzione di test approfonditi per garantire la funzionalità corretta e la sicurezza della piattaforma.
    \item Lancio e Promozione: Lancio ufficiale della piattaforma, promozione attraverso canali di comunicazione e supporto agli utenti durante il primo periodo operativo.
\end{enumerate}

\section*{Scadenze e Milestone:}
\begin{itemize}
    \item Fine del Mese 3: Progettazione completata dell'architettura del marketplace.
    \item Fine del Mese 5: Integrazione e implementazione di tutte le pagine web principali e delle funzionalità chiave.
    \item Fine del Mese 6: Test completati e piattaforma pronta per il lancio.
    \item Fine del Mese 7: Lancio ufficiale del marketplace energetico.
\end{itemize}


\section*{Criteri di successo}
\begin{itemize}
    \item Sottoscrizione di almeno 5000 contratti energetici nel primo anno di attività.
    \item Installazioni di almeno 1000 nuovi dispositivi di fonti rinnovabili nel primo anno di operatività.
    \item Raggiungimento del 40\% di adesione di utility di energia (venditori) della regione d'interesse entro i primi 5 anni dal lancio.
    \item Riduzione delle emissioni di CO2 associate alla produzione e al consumo di energia di almeno il 10\% entro  entro i primi 2 anni di operatività.
\end{itemize}


\section*{Assunzioni / Rischi / Ostacoli}
\begin{itemize}
    \item Si assume che sia disponibile un'infrastruttura tecnologica stabile e affidabile per supportare l'implementazione e l'operatività di TrackER.
    \item Si assume che i venditori di energia mostrino interesse e adesione al concetto di marketplace energetico come TrackER, contribuendo a una base solida di offerte energetiche nel sistema.
    \item Si assume che ci sia una domanda sufficiente di energia da parte degli utenti consumatori che desiderano sfruttare il marketplace di TrackER per le proprie esigenze energetiche.
    \item Rischio che i venditori di energia e gli utenti consumatori siano riluttanti a adottare una nuova piattaforma e ad abbandonare i loro metodi tradizionali di vendita e gestione dell'energia.
    \item Potrebbe esserci una concorrenza da parte di altri marketplace energetici esistenti o nuovi che potrebbero offrire servizi simili o vantaggi competitivi.
    \item Gli ostacoli normativi e le restrizioni legali nel settore energetico possono influenzare la creazione e l'operatività di TrackER. È necessario assicurarsi di rispettare le leggi locali e le normative energetiche vigenti.
    \item Potrebbe essere un ostacolo ottenere l'adesione e la collaborazione di attori chiave del settore energetico, come produttori di energia rinnovabile, fornitori di servizi energetici e regolatori.
    \item L'implementazione di TrackER potrebbe comportare sfide tecniche, come l'integrazione di diverse fonti di dati energetici, la gestione delle transazioni finanziarie e la scalabilità del sistema.
\end{itemize}



\section*{Comunicazioni e Parti Interessate:}
Le parti interessate includono produttori di energia rinnovabile, consumatori, fornitori, esperti di sicurezza, team di sviluppo, consulenti legali e team di marketing. Le comunicazioni saranno gestite attraverso riunioni periodiche, report di avanzamento e canali di comunicazione online.


\end{document}